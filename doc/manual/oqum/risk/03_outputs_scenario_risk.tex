The Scenario Risk Calculator produces the following set of output files:

\begin{enumerate}

  \item \Verb+Aggregate Asset Losses+: this file contains the aggregated scenario
    loss statistics for the entire portfolio of \glspl{asset} defined
    in the \gls{exposuremodel}. The mean (\Verb+mean+) and standard
    deviation (\Verb+stddev+) of the total loss for the portfolio of
    \glspl{asset} are listed in this file.

  \item \Verb+Average Asset Losses+: this file contains mean (\Verb+mean+) and
    associated standard deviation (\Verb+stddev+) of the scenario loss for all
    \glspl{asset} at each of the unique locations in the \gls{exposuremodel}.

  \item \Verb+Aggregate Event Losses+: this file contains the total loss for the
    portfolio of \glspl{asset} defined in the \gls{exposuremodel} for each
    realization of the scenario generated in the Monte Carlo simulation process.

\end{enumerate}

In addition, if the OpenQuake-QGIS \gls{acr:irmt} plugin is used for
visualizing or exporting the results from a Scenario Risk Calculation, the
following additional outputs can be exported:

\begin{enumerate}
\setcounter{enumi}{3}

  \item \Verb+losses_by_tag+: this file contains the scenario
    loss statistics for each of the \glspl{tag} defined in the
    \gls{exposuremodel}. For each \gls{tag}, the mean (\Verb+mean+)
    and associated standard deviation (\Verb+stddev+)
    of the losses for each tag are listed in this file.

\end{enumerate}

If the calculation involves multiple \glspl{acr:gmpe}, separate output files
are generated for each of the above outputs, for each of the different
\glspl{acr:gmpe} used in the calculation.

These different output files for Scenario Risk calculations are described in
more detail in the following subsections.


\subsection{Scenario loss statistics}
\label{subsec:scenario_loss_statistics}

\subsubsection{Asset loss statistics}
\label{subsubsec:scenario_asset_loss_statistics}

This output is always produced for a Scenario Risk calculation and comprises a
mean total loss and associated standard deviation for each of the individual
\glspl{asset} defined in the \gls{exposuremodel} that fall within the
\Verb+region+ and have a computed \gls{acr:gmf} value available
within the defined \Verb+asset_hazard_distance+. These results are stored in a
comma separate value (.csv) file as illustrated in the example shown in
Table~\ref{output:scenario_loss_asset}.

\input{oqum/risk/verbatim/output_scenario_loss_asset}


\subsubsection{Tag loss statistics}
\label{subsubsec:scenario_tag_loss_statistics}

If the OpenQuake-QGIS \gls{acr:irmt} plugin is used for visualizing or
exporting the results from a Scenario Risk Calculation, the total expected
losses for assets of each \gls{tag} will be computed and made available for
export as a csv file. This distribution of losses per asset \gls{tag} is
depicted in the example output file snippet in
Table~\ref{output:scenario_loss_tag}.

\input{oqum/risk/verbatim/output_scenario_loss_tag}

The output file lists the mean loss aggregated for each \glspl{tag}
present in the exposure model and selected by the for all loss types (amongst ``structural'',
``nonstructural'', ``contents'', or ``business\_interruption'') for which a
\gls{vulnerabilitymodel} file was provided in the configuration file.


\subsubsection{Total loss statistics}
\label{subsubsec:scenario_total_loss_statistics}

If the OpenQuake-QGIS \gls{acr:irmt} plugin is used for visualizing or
exporting the results from a Scenario Risk Calculation, the mean total loss
and associated standard deviation for the selected earthquake rupture will be
computed and made available for export as a csv file, as illustrated in the
example shown in Table~\ref{output:scenario_loss_total}.

\input{oqum/risk/verbatim/output_scenario_loss_total}


\subsection{Scenario losses by event}
\label{subsec:scenario_losses_event}

The losses by event output lists the total losses for each realization of the
scenario generated in the Monte Carlo simulation process for all loss types
for which a \gls{vulnerabilitymodel} file was provided in the configuration
file. These results are exported in a comma separate value (.csv) file as
illustrated in the example shown in Table~\ref{output:scenario_loss_event}.

\input{oqum/risk/verbatim/output_scenario_loss_event}


