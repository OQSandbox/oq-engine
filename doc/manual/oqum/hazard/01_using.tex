This Chapter summarises the structure of the information necessary to define
a PSHA input model to be used with the \glsdesc{acr:oqe}.

Input data for probabilistic based seismic hazard analysis (Classical, Event
based, Disaggregation, and Uniform Hazard Spectra) are organised into:

\begin{itemize}

	\item A general configuration file.

    \item A file describing the Seismic Source System, that is the set of
	initial source models and associated epistemic uncertainties needed to
	model the seismic activity in the region of interest.

    \item A file describing the Ground Motion System, that is the set of
	ground motion prediction equations, per tectonic region type, needed to
	model the ground motion shaking in the region of interest.

\end{itemize}

Figure~\ref{fig:psha_input} summarises the structure of a PSHA input model
for the \glsdesc{acr:oqe} and the relationships between the different files.

\begin{figure}[!ht]
\centering
\includegraphics[width=14cm]{figures/hazard/psha_input_structure.pdf}
\caption{PSHA Input Model structure}
\label{fig:psha_input}
\end{figure}


\section{Defining Logic Trees}
\input{oqum/hazard/01a_logic_trees.tex}
\label{sec:hazard_logic_trees}

\section{The Seismic Source System}
\input{oqum/hazard/01b_seismic_source_system}
\label{sec:seismic_source_system}

\section{The Ground Motion System}
\index{Input!Ground motion system}
\label{sec:ground_motion_system}
\input{oqum/hazard/01c_ground_motion_system}

\section{Configuration file}
\index{Input!Configuration file!Hazard}
\label{sec:hazard_configuration_file}
\input{oqum/hazard/01d_config.tex}