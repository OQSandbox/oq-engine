The Event Based PSHA calculator computes and stores stochastic event sets and
the corresponding ground motion fields. This calculator can also produce
hazard curves and hazard maps exactly in the same way as done using the
Classical PSHA calculator. The inset below shows an example of the list of
results provided by the \gls{acr:oqe} at the end of an event-based PSHA
calculation:

\begin{Verbatim}[frame=single, commandchars=\\\{\}, fontsize=\small]
user@ubuntu:~$ oq engine --lo <calc_id>
id | name
\textcolor{red}{10 | Ground Motion Fields}
11 | Hazard Curves
12 | Hazard Maps
13 | Realizations
\textcolor{blue}{14 | ruptures}
15 | Seismic Source Groups
16 | Uniform Hazard Spectra
\end{Verbatim}


This list in the inset above contains a set of ruptures (in blue) and their
corresponding sets of ground motion fields (in red). Exporting the outputs
from the ruptures will produce, for each realisation, an NRML file containing
a collection of ruptures. An example is shown below in Listing~\ref{lst:output_ruptures}.

\input{oqum/hazard/verbatim/output_ses}
\begin{listing}[htbp]
  \inputminted[firstline=1,firstnumber=1,fontsize=\footnotesize,frame=single,linenos,bgcolor=lightgray]{xml}{oqum/hazard/verbatim/output_ses.xml}
  \caption{Example of NRML file containing a collection of ruptures}
  \label{lst:output_ruptures}
\end{listing}

The text in red shows the part which describes the generated
stochastic event sets and the investigation time covered. Inside the
<SES> tag there is a list of integers (a single integer in this example)
which are unique IDs for the seismic events associated to the rupture.
In general a rupture can occur more than once and the number of events
is given by the multiplicity attribute (in this case 1).

The text in blue emphasises the portion of the text used to describe a
rupture. The information provided describes entirely the geometry of the
rupture as well as its rupturing properties (e.g. rake, magnitude). The
rupture ID is an integer that represents each rupture uniquely: it should
not be confused with the event ID.

The outputs from the \glspl{acr:gmf} can be exported either in the xml or csv
formats. If the \glspl{acr:gmf} are exported in the xml format, the
\gls{acr:oqe} will produce an xml file for each realisation with the
corresponding ground motion fields. Listing~\ref{lst:output_gmf_xml} is an
example of a gmf collection NRML file containing one ground motion field:

\begin{listing}[htbp]
  \inputminted[firstline=1,firstnumber=1,fontsize=\footnotesize,frame=single,linenos,bgcolor=lightgray]{xml}{oqum/hazard/verbatim/output_gmf.xml}
  \caption{Example ground motion field collection output file comprising a single GMF}
  \label{lst:output_gmf_xml}
\end{listing}

Exporting the outputs from the \glspl{acr:gmf} in the csv format results in
two csv files illustrated in the example files in
Table~\ref{output:gmf_event_based} and Table~\ref{output:sitemesh}. The sites csv
file provides the association between the site ids in the \glspl{acr:gmf} csv
file with their latitude and longitude coordinates.

\input{oqum/hazard/verbatim/output_gmf_event_based}


The `Seismic Source Groups` output produces a csv file listing the tectonic region
types involved in the calculation and the effective number of ruptures
generated by each of them. An example of such a file is shown below in
Table~\ref{output:event_based_sourcegroups}.

\input{oqum/hazard/verbatim/output_sourcegroups}


The `Realizations` output produces a csv file listing the source model and the
combination of ground shaking intensity models for each path sampled from the
logic tree. An example of such a file is shown below in
Table~\ref{output:realizations}.

\input{oqum/hazard/verbatim/output_realizations}